\documentclass[conference]{IEEEtran}

\hyphenation{op-tical net-works semi-conduc-tor}
\usepackage{graphicx}
\usepackage{lipsum}
\usepackage[table,xcdraw]{xcolor}
\usepackage{caption}
\usepackage{amsmath}

\begin{document}

\title{Visualization of Time \& Time-Oriented Data, A Survey}

\author{\IEEEauthorblockN{Shashank Reddy P \\
500658817}
\IEEEauthorblockA{Ryerson University\\
Toronto, Canada\\
Email: shashank.pandillapal@ryerson.ca}}

\maketitle

\begin{abstract}
%\boldmath
Time is an important dimension with distinct characters and it is common across many application domains like medical, business, science, history etc. The hierarchical structure of time and its natural cycles, re-occurrences make time-oriented data to be treated differently than other kinds of data. It requires appropriate visual and analytical methods to explore and analyze them. In this paper, we survey many visualization techniques dealing with Time-oriented data to categorize and describe tasks the user seek to accomplish using visualization methods. Visualization of Time \& Time-oriented data is a important concern in Visual Analytics, with this survey we try to help users and researchers identify right tasks and techniques to pick for their data in Visual Analytics. \\

\textit{Index Terms} - Time-Oriented Data, Visualization, Analysis, User Information, Graphs, Data Mining, Clustering, Visual Analytics.     
\end{abstract}

\section{Introduction}
The advances in computing and storage technologies have made it possible to create, collect and store huge volumes of data in variety of data formats, languages and cultures\cite{keim}

\section{Result}



\section{Conclusion}
  

\section*{Acknowledgment}



\bibliography{Ref}
\bibliographystyle{ieeetr}

\end{document}


